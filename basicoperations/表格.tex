\documentclass[UTF8]{ctexart}
\begin{document}

\begin{table}[H]
\begin{tabular}{|rrr|}
\hline
直角边 $a$ & 直角边 $b$ & 斜边 $c$\\
\hline
3 & 4 & 5 \\
5 & 12 & 13 \\
\hline
\end{tabular}%
\qquad %2个空格
($a^2 + b^2 = c^2$)
\end{table}

\end{document}

%atex中,由tabular环境完成绘制表格功能,其参数[|rrr|]三个r表示共三列,且为右对齐,两条|表示在第一列前和第三列后有垂直表格线。在tabular环境内部,&用于分割行内各项,即分列,\\用于分行。表格中的横线由\hline绘制。
%表格与插图一样,一般放在table环境里,其使用也大致相同,只是\caption得到的标题为”表“而不是”图“。